\chapter{Demographics}

Honduras is located in Central America where it's bordered by Guatemala, Nicaragua, El Salvador, the Gulf of Honduras, and the Gulf of Fonseca. Honduras was called Spanish Honduras for many years to differentiate it from British Honduras, which later became Belize. Honduras gained independence from Spain in 1821 and has remained independent since. In 2017, Honduras has an estimated population of 9.27 million, which ranks 87th in the world.

The Republic of Honduras has a land mass of 112,492 square kilometres (43,278 square miles) making it the 102nd largest country with regards to surface area alone. For every square kilometre of Honduran territory, there is an average of 64 people here which equates to 166 per square mile and makes Honduras the 128th most densely populated country in the world today.

The capital and largest city of Haiti is Tegucigalpa, commonly called Tegus. Tegus is the administrative and political center of the country . The city's infrastructure has not kept up with the rapid growth, which has led to widespread poverty and condensed living. Tegus has a population of about 1.12 million with a density of 5,600 people per square kilometer (14,516/square mile).

Other major cities include San Pedro Sula (pop: 639,000), Choloma (223,000), La Ceiba (174,000).
Honduras Demographics

The vast majority of the population of Honduras consider themselves to be white or Mestizo. About 90 of the population is Mestizo (mixed European and Amerindian ancestry). In addition, 7 are of American Indian descent while 2 declare themselves to be black.

Most Honduras expatriates are in the United States. It's estimated that 800,000 to 1 million Hondurans are in the U.S., or almost 15 of the country's domestic population.

While most Hondurans are Roman Catholic, the number of people who are members of the Roman Catholic Church has been declining for years while membership in several Protestant churches has increased. A 2008 poll found that 51 of Hondurans are Catholic while 36 are Protestant. About 11 do not have a religious affiliation while 1 claim another religion, such as Buddhism or Islam.
